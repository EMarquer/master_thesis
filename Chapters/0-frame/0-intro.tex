\section{Introduction}
%\todo{speak of neuro-symbolic approaches?}
In recent years there has been an increasing interest in approaches to combine formal knowledge and \textit{artificial neural networks} (NNs), called \textit{neuro-symbolic} approaches~\cite{neuro-symbolic-survey:2017:many-authors,neuro-symbolic-computing:2019:many-authors}.
%Theses approaches, called \textit{neuro-symbolic} approaches, can be implemented along different axes, as mentioned in~\cite{neuro-symbolic-survey:2017:many-authors,neuro-symbolic-computing:2019:many-authors}.
%One way to combine is by extracting knowledge using neral networks \cite[p. 15]{neuro-symbolic-computing:2019:many-authors}
%As mentioned in~\cite[p. 6]{neuro-symbolic-survey:2017:many-authors}, these approaches have a ``generally hierarchical organization'', with the ``lowest-level network [taking] raw data as input and [producing] a model of the dataset''. These networks represent data as real-valued vectors called \textit{embeddings}.
%One of theses axes is to integrate formal knowledge extraction with NNs.
%
%\todo{speak of FCA (and its uses)}
%\todo{Strange}While most neuro-symbolic approaches described in~\cite{neuro-symbolic-survey:2017:many-authors,neuro-symbolic-computing:2019:many-authors} focus on formal logic, \textit{formal concept analysis} (FCA) is another powerful tool for understanding complex data.
\textit{Formal concept analysis} (FCA) is a powerful  formal tool for understanding complex data called \textit{formal context} (FC) (see \cref{sec:fc}).
FCA can be used to generate a structured view of the data, typically a hierarchy of \textit{formal concepts} called \textit{concept lattice} (see \cref{sec:lattice}) or an ontology. %It is used, \eg, to identify specific markers on genes from medical data or to extract lexico-semantic knowledge from corpora.
It can also discover implications between some aspects of the data and generate explainable formal rules grounded on the data.
This can be used to construct decision systems from the data.

%\todo{Justify project}
Replicating FCA's mechanisms using NNs could help processing complex and large datasets~\cite{fca2vec:2019:durrschnabel,lattice-based-nn:2017:kuznetsov} by tackling FCA's scalability issues~\cite{comparing-fca-algorithms:2010:kuznetov}.
%It could also help integrate FCA into connectionist pipelines, \eg, to predict relations between objects or to extract a meaningful subset of rules from large dataset.
%Finally, it may allow us to discover new aspects that the standard FCA process doesn't explore.
It could also help integrate FCA into connectionist pipelines and allow us to discover new aspects that the standard FCA process doesn't explore.
Following this idea, we want to reproduce the general discovery process of FCA with NN architectures.
This asks for a general framework capable of generating concept lattices using exclusively the FC.

%\todo{Explain scarcity of neuro-symbolic approach in FCA}
%For example~\cite{clustering-bipartie-closure:2013:gaume} applies graph clustering on formal contexts by representing them as bipartite graphs.
To our knowledge, there are only a few neuro-symbolic approaches involving FCA.
Rudolph \etal{}~\cite{encoding:2007:rudolph} show how to encode closure operators with simple feed-forward NNs.
Gaume \etal{}~\cite{clustering-bipartie-closure:2013:gaume} make the parallel between FCA and bipartite graph analysis, by considering the FC as a bipartite graph. %However, this approach does not involve deep learning, even if it offers an alternative set of tools to reproduce FCA.
In~\cite{lattice-based-nn:2017:kuznetsov}, Kuznetsov \etal{} present an approach to construct NNs from the results of FCA.
More recently, Dürrschnabel \etal{}~\cite{fca2vec:2019:durrschnabel} present FCA2VEC, a framework to represent FCs by encoding FCA's closure operators, using the results of~\cite{encoding:2007:rudolph}.
However, those approaches do not offer a complete neural framework to approximate FCA.

% contributions
To overcome these limitations, we explore two complementary approaches to the generation of concept lattices using NNs, respectively the generation of lattices as labeled graphs and the generation of concept intents.
Our work provides a framework for the task of reproducing FCA using NNs, supported by experimental results that serve as a baseline.
We also provide a representation framework for FCs called \textit{Bag of Attributes} (BoA) that we detail in~\cite{boa:2020:marquer}.
%\todo[inline]{insert practical example here?}
%The two approaches, respectively the generation of lattices as labeled graphs and the generation of intents, are described and justified in \cref{sec:problem}.

%\todo{Be more precise}
This report is organized as follows.
\cref{ch:frame} introduces the subject and the background of the project. % including a description of our work environment, some background knowledge and a description of the project goals.
In \cref{sec:lab} we present the Inria Project Lab HyAIAI and the LORIA laboratory, as well as our main development tools. In \cref{sec:background_fca} and \cref{sec:background_nn} we introduce basic notions respectively of FCA and deep learning.
Finally, we explain the goals of the project in \cref{sec:problem}.
%
We describe the two approaches we explored respectively in \cref{ch:graph} and \cref{ch:intents}, by presenting a detailed overview of the literature, a description of the NN architectures we tested and the experiments we ran.
\cref{ch:graph} contains an overview of the \soa{} of graph generation, a description of our data representation and generation process, and the details of the architecture we developed in our initial approach based on graphs.
The challenges encountered when designing this architecture led us to create BoA, an embedding framework to represent FCs.
BoA is detailed in \cref{ch:boa} together with FCA2VEC~\cite{fca2vec:2019:durrschnabel}.
\cref{ch:intents} is the description of the second approach we explored, a very modular approach to the problem of lattice generation focused on intents.
Finally, we discuss the results of the project and explain some plans to further expand our approach in \cref{ch:ccl}.